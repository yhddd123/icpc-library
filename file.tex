\documentclass[12pt]{ctexart}
\usepackage{amsfonts,amssymb}
\usepackage{booktabs}
\usepackage{graphicx}
\usepackage{amsmath}
\usepackage{listings}
\usepackage{geometry}
\usepackage[usenames,dvipsnames]{xcolor}
\usepackage{setspace}
\usepackage{hyperref}
\usepackage{float}
\setstretch{1.1} 
\geometry{a4paper}
\geometry{top=2cm} 
\geometry{bottom=1cm} 
\geometry{left=1.5cm, right=1.5cm}
\tolerance=1000
\definecolor{mygreen}{rgb}{0,0.6,0}
\definecolor{mygray}{rgb}{0.5,0.5,0.5}
\definecolor{mymauve}{rgb}{0.58,0,0.82}
\lstset{ %
backgroundcolor=\color{white},   % choose the background color
basicstyle=\footnotesize\ttfamily,        % size of fonts used for the code
columns=fullflexible,
inputencoding=utf8,
extendedchars=true, 
breaklines=true,                 % automatic line breaking only at whitespace
captionpos=b,                    % sets the caption-position to bottom
tabsize=4,
commentstyle=\color{mygreen},    % comment style
escapeinside={\%*}{*)},          % if you want to add LaTeX within your code
keywordstyle=\color{blue},       % keyword style
stringstyle=\color{mymauve}\ttfamily,     % string literal style
frame=single,
rulesepcolor=\color{red!20!green!20!blue!20},
% identifierstyle=\color{red},
language=c++,
}

\title{yhddd 的 icpc 模板}
\author{yhddd}
\date{\today}

\begin{document}

\maketitle

\centerline{ver: 1.0.0}

\newpage

\tableofcontents

\newpage

\section{杂项}

\subsection{fastio}

\begin{lstlisting}
    static char buf[1000000],*p1=buf,*p2=buf;
    #define getchar() p1==p2&&(p2=(p1=buf)+fread(buf,1,1000000,stdin),p1==p2)?EOF:*p1++
    inline int read(){int x=0,f=1;char c=getchar();while(c<'0'||c>'9'){if(c=='-')f=-1;c=getchar();}while(c>='0'&&c<='9'){x=(x<<3)+(x<<1)+c-48;c=getchar();}return x*f;}
    inline void write(int x){static char buf[20];static int len=-1;if(x<0)putchar('-'),x=-x;do buf[++len]=x%10,x/=10;while(x);while(len>=0)putchar(buf[len--]+48);}
    #define put() putchar(' ')
    #define endl puts("")
\end{lstlisting}

\subsection{取模优化}

\subsection{i24}

\newpage

\section{dp}

\subsection{决策单调性}

\subsubsection{SMAWK}

求 $n\times m$ 矩阵每行最小值。矩阵满足对于任意 $x<y$,$x$ 行最小值位置 $\le$ $y$ 行最小值位置。

cf 的交互格式,没有实际用过。

$4(n+m)$ 次。

\begin{lstlisting}
int n,m,a[10][10];
map<pii,int> mp;
int ask(int u,int v){
    if(mp.find({u,v})!=mp.end())return mp[{u,v}];
    printf("? %d %d\n",u,v);fflush(stdout);
    // return mp[{u,v}]=a[u][v];
    return mp[{u,v}]=read();
}
int pre[maxn],suf[maxn];
vector<int> reduce(vector<int> x,vector<int> y){
    int n=x.size(),m=y.size();
    for(int i=0;i<m-1;i++)pre[y[i+1]]=y[i],suf[y[i]]=y[i+1];
    pre[y[0]]=0,suf[0]=y[0],suf[y[m-1]]=y[m-1];
    auto del=[&](int u){suf[pre[u]]=suf[u],pre[suf[u]]=pre[u];};
    for(int i=0,j=y[0],t=n+1;t<=m;){
        if(ask(x[i],j)>ask(x[i],suf[j])){
            del(j);t++;
            if(i)i--,j=pre[j];
            else j=suf[j];
        }
        else{
            if(i==n-1)del(suf[j]),t++;
            else i++,j=suf[j];
        }
    }
    y.clear();
    for(int i=0,j=suf[0];i<n;i++,j=suf[j])y.pb(j);
    return y;
}
int p[maxn];
void smawk(vector<int> x,vector<int> y){
    y=reduce(x,y);
    if(x.size()==1){
        p[x[0]]=y[0];
        return ;
    }
    vector<int> z;
    for(int i=0;i<x.size();i++)if(!(i&1))z.pb(x[i]);
    smawk(z,y);
    for(int i=1;i<x.size();i+=2){
        int l=lower_bound(y.begin(),y.end(),p[x[i-1]])-y.begin();
        int r=(i==x.size()-1?y.size()-1:lower_bound(y.begin(),y.end(),p[x[i+1]])-y.begin());
        p[x[i]]=y[l];
        for(int j=l+1;j<=r;j++)if(ask(x[i],y[j])<ask(x[i],p[x[i]]))p[x[i]]=y[j];
    }
}
void work(){
    n=read();m=read();
    // for(int i=1;i<=n;i++){
    //     for(int j=1;j<=n;j++)a[i][j]=read();
    // }
    vector<int> x(n),y(m);
    for(int i=1;i<=n;i++)x[i-1]=i;
    for(int i=1;i<=m;i++)y[i-1]=i;
    smawk(x,y);
    int ans=inf;for(int i=1;i<=n;i++)ans=min(ans,ask(i,p[i]));
    // for(int i=1;i<=n;i++)cout<<p[i]<<" ";cout<<"\n";
    printf("! %d\n",ans);fflush(stdout);
}
\end{lstlisting}

\subsubsection{LARSCH}

基于魔改的分治,可以在线, $O(n\log n)$,支持类莫队计算贡献,常数小,码量小。

\begin{lstlisting}
    
struct ds{
    int l,r,ans;
    ds(){l=1,r=0;}
    ll que(int ql,int qr){
        while(r<qr)r++;
        while(l>ql)l--;
        while(r>qr)r--;
        while(l<ql)l++;
        return ans;
    }
}a[2];
void upd(int j,int i,int op){
	int nw=dp[j-1]+a[op].que(j,i)+w;
	if(nw<dp[i])dp[i]=nw,p[i]=j;
}
void sovle(int l,int r){
	if(l==r)return ;
	int mid=l+r>>1;
	for(int i=p[l];i<=p[r];i++)upd(i,mid,0);
	sovle(l,mid);
	for(int i=l;i<=mid;i++)upd(i,r,1);
	sovle(mid+1,r);
}
\end{lstlisting}

\section{ds}

\subsection{线段树}

\subsubsection{zkw}

\subsubsection{楼房重建}

\begin{lstlisting}
#define mid ((l+r)>>1)
#define ls nd<<1
#define rs nd<<1|1
int mn[maxn<<2],ans[maxn<<2];
int query(int nd,int l,int r,int w){
	if(w<mn[nd])return (r-l+1)*w;
	if(l==r)return mn[nd];
	if(mn[ls]<=w)return query(ls,l,mid,w)+ans[rs];
	else return (mid-l+1)*w+query(rs,mid+1,r,w);
}
void up(int nd,int l,int r){
	mn[nd]=min(mn[ls],mn[rs]);
	ans[rs]=query(rs,mid+1,r,mn[ls]);
}
void build(int nd,int l,int r){
	if(l==r){
		mn[nd]=a[l];
		return ;
	}
	build(ls,l,mid),build(rs,mid+1,r);
	up(nd,l,r);
}
int query(int nd,int l,int r,int ql,int qr,int &w){
	// cout<<l<<" "<<r<<" "<<ql<<" "<<qr<<" "<<mn[nd]<<" "<<w<<"\n";
	if(l>=ql&&r<=qr){
		int res=query(nd,l,r,w);
		w=min(w,mn[nd]);
		return res;
	}
	int res=0;
	if(ql<=mid)res+=query(ls,l,mid,ql,qr,w);
	if(qr>mid)res+=query(rs,mid+1,r,ql,qr,w);
	return res;
}
\end{lstlisting}

\subsubsection{beats}

\subsubsection{分裂}

\begin{lstlisting}
int merge(int u,int v,int l,int r){
	if(!u||!v)return u|v;
	if(l==r){tree[u]+=tree[v];clr(v);return u;}
	lc[u]=merge(lc[u],lc[v],l,mid);
	rc[u]=merge(rc[u],rc[v],mid+1,r);
	tree[u]=tree[lc[u]]+tree[rc[u]];clr(v);
	return u;
}
int split(int nd,int l,int r,ll k){
	if(!nd)return 0;
	int u=newnode();
	if(k>tree[ls])rc[u]=split(rs,mid+1,r,k-tree[ls]);
	else rc[u]=rs,rs=0;
	if(k<tree[ls])lc[u]=split(ls,l,mid,k);
	tree[nd]=tree[ls]+tree[rs],tree[u]=tree[lc[u]]+tree[rc[u]];
	return u;
}
\end{lstlisting}

\subsubsection{KTT}

\subsection{平衡树}

\subsection{莫队}

\subsection{分块}

\subsection{手写 STL}

\subsubsection{bitset}

\begin{lstlisting}
#define ull unsigned long long
ull pw[65];
struct bs{
	vector<ull> a;
	int len,n;
	void init(int _n){
		n=_n,len=(n+63)/64;a.resize(len+1,0);
	}
	void set0(int x){a[x>>6]&=~pw[x&63];}
	void set1(int x){a[x>>6]|=pw[x&63];}
	bool operator[](int x){return (a[x>>6]>>(x&63))&1;}
	bs operator|(const bs&b)const{
		bs c;c.init(max(n,b.n));
		for(int i=0;i<c.len;i++)c.a[i]=a[i]|b.a[i];
		return c;
	}
	bs operator&(const bs&b)const{
		bs c;c.init(min(n,b.n));
		for(int i=0;i<c.len;i++)c.a[i]=a[i]&b.a[i];
		return c;
	}
	void operator|=(const bs&b){
		for(int i=0;i<max(len,b.len);i++)a[i]|=b.a[i];
	}
	void operator&=(const bs&b){
		for(int i=0;i<min(len,b.len);i++)a[i]&=b.a[i];
	}
	bs operator<<(int x)const{
		bs res;res.init(n);
		int y=x>>6,z=x&63;
		ull lst=0;
		for(int i=0;i+y<res.len;i++){
			res.a[i+y]=lst|(a[i]<<z);
			if(z)lst=a[i]>>(64ll-z);
		}
		return res;
	}
	int count(){
		int res=0;for(int i=0;i<len;i++)res+=__builtin_popcountll(a[i]);
		return res;
	}
}f;
\end{lstlisting}

\subsubsection{哈希表}

\begin{lstlisting}
struct hsh_table{
	int head[maxn],tot;
	struct nd{
		int nxt;
		ull key;
		int val;
	}e[maxn];
	inline int hsh(ull u){return u%maxn;}
	bool find(ull key){
		int u=hsh(key);
		for(int i=head[u];i;i=e[i].nxt){
			if(e[i].key==key)return 1;
		}
		return 0;
	}
	inline int &operator[](ull key){
		int u=hsh(key);
		for(int i=head[u];i;i=e[i].nxt){
			if(e[i].key==key)return e[i].val;
		}
		e[++tot]={head[u],key,0};head[u]=tot;
		return e[tot].val;
	}
	void clear(){
		tot=0;
		for(int i=0;i<maxn;i++)head[i]=0;
	}
}mp;
\end{lstlisting}

\newpage

\section{graph}

\subsection{树}

\subsubsection{毛毛虫剖分}

\begin{lstlisting}
int n,q,k=3;
vector<int> e[maxn];
int siz[maxn],son[maxn],fa[maxn];
int dfn[maxn],st[18][maxn],tim;
void dfs(int u){
	siz[u]=1;
	st[0][dfn[u]=++tim]=fa[u];
	vector<int> tmp;
	for(int v:e[u])if(v!=fa[u]){
		fa[v]=u;dfs(v);siz[u]+=siz[v];
		tmp.pb(v);
	}
	e[u]=tmp;
	sort(e[u].begin(),e[u].end(),[&](int u,int v){return siz[u]>siz[v];});
	if(e[u].size())son[u]=e[u][0];	
}
int mmax(int u,int v){return dfn[u]<dfn[v]?u:v;}
void lca_init(){
	for(int j=1;j<18;j++){
		for(int i=1;i+(1<<j)-1<=n;i++)st[j][i]=mmax(st[j-1][i],st[j-1][i+(1<<j-1)]);
	}
}
int lca(int u,int v){
	if(u==v)return u;
	u=dfn[u],v=dfn[v];if(u>v)swap(u,v);u++;
	int k=__lg(v-u+1);
	return mmax(st[k][u],st[k][v-(1<<k)+1]);
}
int id[maxn],idx;
void downid(int u,int d){
	if(!d){
		if(!id[u])id[u]=++idx;
		return ;
	}
	for(int v:e[u])downid(v,d-1);
}
void dfsid(int u){
	vector<int> path;
	for(int x=u;x;x=son[x])path.pb(x);
	for(int i=0;i<=k;i++){
		for(int u:path)downid(u,i);
	}
	reverse(path.begin(),path.end());
	for(int u:path){
		for(int v:e[u])if(v!=son[u])dfsid(v);
	}
}
void merge(vector<pii> &u,vector<pii> v){
	for(auto p:v)u.pb(p);
}
void reinit(vector<pii> &u){
	sort(u.begin(),u.end());
	vector<pii> nw;
	for(auto[l,r]:u){
		if(nw.size()&&nw.back().se+1==l)nw.back().se=r;
		else nw.pb({l,r});
	}
	u=nw;
}
vector<pii> sub[maxn],kson[maxn][maxk],bro[maxn][maxk];
void dfsup(int u){
	sub[u]={{id[u],id[u]}},kson[u][0]={{id[u],id[u]}};
	for(int v:e[u]){
		dfsup(v);
		merge(sub[u],sub[v]);
		for(int i=0;i<=k;i++)bro[v][i]=kson[u][i];
		for(int i=1;i<=k;i++)merge(kson[u][i],kson[v][i-1]),reinit(kson[u][i]);
	}
	if(e[u].size()){
		vector<pii> tmp[maxk];
		for(int ii=e[u].size()-1;~ii;ii--){
			int v=e[u][ii];
			for(int i=1;i<=k;i++)merge(bro[v][i],tmp[i]),reinit(bro[v][i]);
			for(int i=1;i<=k;i++)merge(tmp[i],kson[v][i-1]),reinit(tmp[i]);
		}
	}
	reinit(sub[u]);
}
vector<pii> kans[maxn][maxk],kdis[maxn][maxk];
void dfsdw(int u){
	for(int i=0;i<=k;i++){
		kans[u][i]=kans[fa[u]][i];
		merge(kans[u][i],kson[u][i]);
		reinit(kans[u][i]);
	}
	for(int i=0;i<=k;i++){
		for(int j=0;j<=i;j++)merge(kdis[u][i],kson[u][j]);
		for(int j=1,x=u;j<=k&&x;x=fa[x],j++){
			for(int k=0;k<=i-j;k++)merge(kdis[u][i],bro[x][k]);
		}
		reinit(kdis[u][i]);
	}
	for(int v:e[u])dfsdw(v);
}
vector<pii> getsub(int u){return sub[u];}
vector<pii> gettp(int u,int tp,int k){
	vector<pii> a=kans[u][k],b=kans[tp][k],nw;
	// cout<<u<<" "<<tp<<" "<<k<<" "<<a.size()<<" "<<b.size()<<endl;
	for(int i=0,l=0;i<a.size();i++){
		while(l<b.size()&&b[l].se<a[i].fi)l++;
		int r=l;while(r<b.size()&&b[r].se<=a[i].se)r++;
		if(l==r)nw.pb(a[i]);
		else{
			int lst=a[i].fi;
			for(int j=l;j<r;j++){
				if(lst<b[j].fi)nw.pb({lst,b[j].fi-1});
				lst=b[j].se+1;
			}
			if(lst<=a[i].se)nw.pb({lst,a[i].se});
		}
		l=r;
	}
	reinit(nw);
	return nw;
}
vector<pii> getpath(int u,int v,int k){
	int tp=lca(u,v);
	vector<pii> a=kdis[tp][k];
	merge(a,gettp(u,tp,k));
	merge(a,gettp(v,tp,k));
	reinit(a);
	return a;
}
void work(){
	n=read();q=read();
	for(int i=1;i<n;i++){
		int u=read(),v=read();
		e[u].pb(v),e[v].pb(u);
	}
	dfs(1);
	lca_init();
	dfsid(1);
	dfsup(1);
	dfsdw(1);
	build(1,1,n);
	while(q--){
		int op=read();
		if(op==1){
			int u=read(),v=read(),k=read(),w=read();
			vector<pii> a=getpath(u,v,k);
			for(auto[l,r]:a)updata(1,1,n,l,r,w);
		}
		if(op==2){
			int u=read(),w=read();
			vector<pii> a=getsub(u);
			for(auto[l,r]:a)updata(1,1,n,l,r,w);
		}
		if(op==3){
			int u=read(),v=read(),k=read();
			vector<pii> a=getpath(u,v,k);
			ll ans=0;for(auto[l,r]:a)ans+=quesum(1,1,n,l,r);
			printf("%lld\n",ans);
		}
		if(op==4){
			int u=read();
			vector<pii> a=getsub(u);
			ll ans=0;for(auto[l,r]:a)ans+=quesum(1,1,n,l,r);
			printf("%lld\n",ans);
		}
		if(op==5){
			int u=read(),v=read(),k=read();
			vector<pii> a=getpath(u,v,k);
			ll ans=-inf;for(auto[l,r]:a)ans=max(ans,quemx(1,1,n,l,r));	
			printf("%lld\n",ans);
		}
		if(op==6){
			int u=read();
			vector<pii> a=getsub(u);
			ll ans=-inf;for(auto[l,r]:a)ans=max(ans,quemx(1,1,n,l,r));	
			printf("%lld\n",ans);
		}
	}
}
\end{lstlisting}

\subsubsection{点分治}

\subsubsection{动态 dp}

\subsubsection{purfer 序}

\subsection{连通性}

\subsubsection{边双}

\begin{lstlisting}
int dfn[maxn],lw[maxn],idx;
int st[maxn],tp;
vector<int> g[maxn];
int scct;
bool vis[maxn];
void tar(int u,int fl){
    dfn[u]=lw[u]=++idx;st[++tp]=u;
    for(int i=head[u];i;i=e[i].nxt){
        int v=e[i].to;
        if(i==(fl^1))continue;
        if(!dfn[v]){
            tar(v,i);
            lw[u]=min(lw[u],lw[v]);
        }
        else lw[u]=min(lw[u],dfn[v]);
    }
    if(lw[u]==dfn[u]){
        g[++scct].push_back(st[tp]);
        while(st[tp--]!=u){
            g[scct].push_back(st[tp]);
        }
    }
}
\end{lstlisting}

\subsubsection{点双}

\begin{lstlisting}
vector<int> e[maxn],g[maxn];
int dfn[maxn],idx,lw[maxn];
int st[maxn],tp;
void tar(int u){
    dfn[u]=lw[u]=++idx;st[++tp]=u;
    for(int v:e[u]){
        if(!dfn[v]){
            tar(v);
            lw[u]=min(lw[u],lw[v]);
            if(lw[v]>=dfn[u]){
                g[++num].push_back(st[tp]);
                while(st[tp--]!=v){
                    g[num].push_back(st[tp]);
                }
                g[num].push_back(u);
            }
        }
        else lw[u]=min(lw[u],dfn[v]);
    }
}
\end{lstlisting}

\subsubsection{双极定向}

\begin{lstlisting}
int n,m,s,t;
int head[maxn],tot;
struct nd{
	int nxt,to;
}e[maxn<<1];
void add(int u,int v){e[++tot]={head[u],v};head[u]=tot;}
pii g[maxn];
int lw[maxn],dfn[maxn],idx,fa[maxn];
vector<int> id;
bool vis[maxn];
bool dfs(int u){
	dfn[u]=lw[u]=++idx;vis[u]=1;
	bool fl=u==t;
	for(int i=head[u];i;i=e[i].nxt){
		int v=e[i].to;
		if(!vis[v]){
			fa[v]=u;fl|=dfs(v);
			lw[u]=min(lw[u],lw[v]);
		}
		else lw[u]=min(lw[u],dfn[v]);
	}
	if(fl)id.pb(u);
	return fl;
}
queue<int> q;
int d[maxn];
vector<int> a[maxn];
int st[maxn],tp,rnk[maxn];
void dfs1(int u){
	if(vis[u])return ;vis[u]=1;
	st[++tp]=u;
	for(int v:a[dfn[u]])dfs1(v);
}
void work(){
	n=read();m=read();s=read();t=read();
	for(int i=1;i<=n;i++)head[i]=0;tot=0;
	for(int i=1;i<=m;i++){
		int u=read(),v=read();
		add(u,v),add(v,u);
		g[i]={u,v};
	}
	idx=0;id.clear();
	for(int i=1;i<=n;i++)vis[i]=0;
	fa[s]=0;dfs(s);
	for(int i=1;i<=n;i++)d[i]=0;
	for(int i:id)d[i]++;
	for(int i=1;i<=n;i++)d[fa[i]]++;
	for(int i=1;i<=n;i++)if(!d[i])q.push(i);
	for(int i=1;i<=n;i++)a[i].clear();
	while(!q.empty()){
		int u=q.front();q.pop();
		a[lw[u]].pb(u),a[dfn[fa[u]]].pb(u);
		d[fa[u]]--;
		if(!d[fa[u]])q.push(fa[u]);
	}
	tp=0;
	for(int i=1;i<=n;i++)vis[i]=0;
	while(id.size())dfs1(id.back()),id.pop_back();
	if(st[1]!=s||st[tp]!=t){puts("No");return ;}
	if(tp!=n){puts("No");return ;}
	for(int i=1;i<=n;i++)vis[i]=0;
	vis[st[1]]=1;
	for(int i=2;i<=tp;i++){
		int u=st[i];bool fl=0;
		for(int i=head[u];i;i=e[i].nxt){
			int v=e[i].to;
			fl|=vis[v];
		}
		if(!fl){puts("No");return ;}
		vis[u]=1;
	}
	for(int i=1;i<=n;i++)vis[i]=0;
	vis[st[tp]]=1;
	for(int i=tp-1;i;i--){
		int u=st[i];bool fl=0;
		for(int i=head[u];i;i=e[i].nxt){
			int v=e[i].to;
			fl|=vis[v];
		}
		if(!fl){puts("No");return ;}
		vis[u]=1;
	}
	for(int i=1;i<=n;i++)rnk[st[i]]=i;
	puts("Yes");
	for(int i=1;i<=m;i++){
		auto[u,v]=g[i];
		if(rnk[u]>rnk[v])swap(u,v);
		printf("%lld %lld\n",u,v);
	}
}
\end{lstlisting}

\subsubsection{广义串并联图}

\begin{lstlisting}
map<int,int> mp[maxn];
int d[maxn];
queue<int> q;
void add(int u,int v,int w){
    if(mp[u].find(v)!=mp[u].end())ans=max(ans,mp[u][v]+w);
    else mp[u][v]=-inf,d[u]++;
    mp[u][v]=max(mp[u][v],w);
}
void work(){
    n=read();m=read();
    for(int i=1;i<=m;i++){
        int u=read(),v=read(),w=read();
        add(u,v,w),add(v,u,w);
    }
    for(int i=1;i<=n;i++)if(d[i]<=2)q.push(i);
    while(!q.empty()){
        int u=q.front();q.pop();
        if(!d[u])continue;
        else if(d[u]==1){
            int v=(*mp[u].begin()).fi;
            mp[u].erase(v),mp[v].erase(u),d[u]--,d[v]--;
            if(d[v]<=2)q.push(v);
        }
        else if(d[u]==2){
            int v1=(*mp[u].begin()).fi,v2=(*--mp[u].end()).fi;
            int w1=(*mp[u].begin()).se,w2=(*--mp[u].end()).se;
            add(v1,v2,w1+w2),add(v2,v1,w1+w2);
            mp[u].erase(v1),mp[u].erase(v2),mp[v1].erase(u),mp[v2].erase(u),d[u]-=2,d[v1]--,d[v2]--;
            if(d[v1]<=2)q.push(v1);
            if(d[v2]<=2)q.push(v2);
        }
    }
    printf("%lld\n",ans);
}
\end{lstlisting}

\subsection{流}

\subsubsection{预留推进}

\subsubsection{原始对偶}

\subsubsection{最小割树}

\subsubsection{一般图最大匹配}

\subsection{杂项}

\subsubsection{欧拉回路}

\subsubsection{四元环计数}

\begin{lstlisting}
vector<int> e[maxn],g[maxn];
int d[maxn],cnt[maxn],ans;
void work(){
    n=read();m=read();
    for(int i=1;i<=m;i++){
        int u=read(),v=read();
        e[u].push_back(v),e[v].push_back(u);
        d[u]++,d[v]++;
    }
    for(int u=1;u<=n;u++){
        for(int v:e[u]){
            if(d[u]>d[v]||(d[u]==d[v]&&u>v))g[u].push_back(v);
        }
    }
    for(int i=1;i<=n;i++){
        for(int j:g[i]){
            for(int k:e[j])if(d[i]>d[k]||(d[i]==d[k]&&i>k))ans+=cnt[k]++;
        }
        for(int j:g[i]){
            for(int k:e[j])cnt[k]=0;
        }
    }
    printf("%lld\n",ans);
}
\end{lstlisting}

\newpage

\section{geometry}

???

\section{math}

\subsection{筛}

\subsection{矩阵}

\subsubsection{高斯消元}

\subsubsection{矩阵求逆}

\subsubsection{行列式}

\subsubsection{特征多项式}

\subsection{poly}

\subsubsection{fft}

\subsubsection{ntt}

\subsubsection{mtt}

\subsubsection{ln exp}

\subsubsection{多点求值}

\subsection{集合幂级数}

\subsubsection{FWT}

\subsubsection{子集卷积}

\subsubsection{多项式复合集合幂级数}

\subsection{插值}

\section{string}


\end{document}